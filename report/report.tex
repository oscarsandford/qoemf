% Based on the template for ICIP-2022 paper; to be used with:
%          spconf.sty  - ICASSP/ICIP LaTeX style file, and
%          IEEEbib.bst - IEEE bibliography style file.
% --------------------------------------------------------------------------
\documentclass{article}
\usepackage{spconf,amsmath,graphicx}

% Example definitions.
% --------------------
\def\x{{\mathbf x}}
\def\L{{\cal L}}


\title{A Simple Load Testing QoS Measurement Framework}
\name{Juan Flores, Oscar Sandford, and Ben Wunderlich
\thanks{Thank you to Dr. Jianping Pan at the University of Victoria for his teaching and guidance throughout this project and the course.}}
\address{University of Victoria}
\begin{document}
%
\maketitle
%
\begin{abstract}
Write this last.
\end{abstract}
%
\begin{keywords}
Quality of Service, Software Defined Network, Network emulation, Mininet
\end{keywords}
%
% Overview of the problem. Why is it important?
\section{Introduction}
The design of computer networks is driven by a drive towards a target quality of service (QoS), a set of performance measurements. Further, the implementation of 
network architectures demands time and money, and implementations that fail to meet quality of service standards waste time for both companies and customers. 
Software-defined networks (SDN) addresses the project of static architectures by enabling programming and measurement in a dynamic setting. It is important to understand 
and harness the strengths of SDN in a methodical way, so developers can focus more on the design goals and less on tedious implementation tasks.

% What has already been done? Why are their efforts not sufficient?
\section{Related Work}
Previous studies have defined quality of services measurements \cite{qos_analysis_2022} and used network emulation tools to program their own experiments 
\cite{chauhan_atulkar_2020}. However, there does not seem to be a general method through which network emulation tools can be used in association with quality 
of service measurement techniques to test network architectures for their performance before they are implemented in real-world applications.

% What is our approach? Why can it do better or differently?
\section{Approach}
Mininet will be utilized as a network emulator, for it has seen significant use in related works \cite{mininet_emulation_2014,qos_analysis_2022,chauhan_atulkar_2020} and 
is well-documented. The Iperf tool was used in \cite{chauhan_atulkar_2020} for generating network traffic, and we think that this tool will be appropriate for the same 
purposes in this study. We will reuse the quality of service metrics outlined in \cite{qos_analysis_2022}: 
\begin{itemize}
    \item Throughput (total transmitted data in bits)/(total time taken in seconds)
    \item Delay (time required to transmit the data from sender to receiver)
    \item Packet loss (the number of packets not delivered to their destination)
    \item Jitter (the variance in latency)
\end{itemize} 
We will use Mininet's toolkit create a server around a given application (i.e. a simple Node.js application), and artificially adjust flow control, packet drop rate, 
and jitter from the server-side. Mininet will be able to measure packet return trip time (RTT), delay, and throughput from the server to clients, and the results of 
these measurements will determine how the overall quality of service changes as various server-side features are tweaked dynamically. Further, we will show how to use 
Mininet to simulate DoS attacks by a large number of clients in order to observe how attacks affect quality of service to benign users. \\\\
In this way, we will provide a quality of service measurement outline for future developers to consider when testing their network applications in a software-driven 
manner. 

% Expected deliverables and a rough biweekly time schedule.
\subsection{Timeline}
\begin{enumerate}
    \item Jan 31st - Feb 13th: Create project outline, gather resources, outline program and create git repository.
    \item Feb 14th - Feb 27th: Build testing environment and prepare test.
    \item Feb 28th - Mar 13th: Begin tests, create midterm presentation and present. 
    \item Mar 14th - Mar 27th: Gather results and prepare final presentation. Begin final report.
    \item Mar 28th - Apr 7th: Final Presentation. Continue to work on final report.
    \item Apr 8th - Apr 16th: Turn in final report.
\end{enumerate}

\subsection{Project Website}
Please see the associated website for progress reports and results at https://oscarsandford.github.io/qoemf/.

% References should be produced using the bibtex program from suitable
% BiBTeX files (here: strings, refs, manuals). The IEEEbib.bst bibliography
% style file from IEEE produces unsorted bibliography list.
% -------------------------------------------------------------------------
\bibliographystyle{IEEEbib}
\bibliography{refs}

\end{document}
